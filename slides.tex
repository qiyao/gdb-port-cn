\documentclass[cjk]{beamer}  % 使用Beamer包
%\hypersetup{pdfpagemode=FullScreen}
\usepackage{CJK}   % 中文环境
\usepackage{indentfirst}
%\usepackage{beamerthemesplit}

\usepackage{makeidx}
%\usepackage[dvipdfm,CJKbookmarks,bookmarks=true,bookmarksnumbered=true]{hyperref}
%\usepackage[CJKbookmarks,bookmarks=true,bookmarksnumbered=true]{hyperref}
\usepackage{listings}


\hypersetup{colorlinks, linkcolor=blue, citecolor=blue,
    urlcolor=blue,
    plainpages=flase,
    pdfcreator=tex,
    bookmarksopen=true,
    pdfhighlight=/P,
    pdfauthor={Yao Qi <qiyaoltc@gmail.com>},
    pdfcreator={cTeX},
%    pdftitle={iptables 入门},
%    pdfkeywords={iptables 入门 Rae szlug},
    pdfstartview=FitH,
    pdfpagemode=UseOutlines,%UseOutlines, %None, FullScreen, UseThumbs
}

%\usepackage[all,bottom]{draftcopy}
%\draftcopyName{Copyright by Rae <crquan@gmail.com>, 2006}{50}
%\draftcopySetGrey{0.8} \draftcopyPageTransform{55 rotate}
%\draftcopyPageX{80}\draftcopyPageY{-25}

\lstset{
  basicstyle=\tiny,          % print whole listing small
  showstringspaces=false,
  numbers=left,
  numberstyle=\tiny,
  % labelstep=1,
  % commentstyle=\textsl,
  keywordstyle=\color{green}\bfseries\underbar,     % underlined bold red keywords
  identifierstyle={},         % nothing happens to other identifiers
  commentstyle=\mdseries\color{blue}, % white comments
  stringstyle=\color{red}\ttfamily}      % typewriter type for strings
% stringspaces=false}         % no special string spaces



%\usetheme{Madrid}  % 采用的主题
\usetheme{Warsaw}
%\usetheme{Rochester}
%\usetheme{Berkeley}
%\usetheme{PaloAlto}
%\usetheme{Montpellier}
%\usetheme{Berlin}
%\usetheme{Ilmenau}
%\usetheme{Dresden}

%\usecolortheme{albatross} % 采用的配色。
%\usecolortheme{beaver} white background
%\usecolortheme{beetle}
%\usecolortheme{dolphin}
%\usecolortheme{fly}
%\usecolortheme{lily}

%\useoutertheme{tree}

\usefonttheme{structureitalicserif}

\begin{document}
\begin{CJK}{UTF8}{gbsn}

\title{移植GDB的一些思路}
\author{\href{mailto:qiyaoltc@gmail.com}{齐尧}}
\date{\today}

% 生成上面定义的"\title", "\author", "\date"等信息
\frame{\titlepage}


% 生成目录菜单
\section{目录及流程}
\frame{
    \tableofcontents
}

\section{Two kinds of GDB}

\begin{frame}
  \frametitle{Two parts of GDB}
  \begin{columns}
    \column{5cm}
    \begin{overprint}

      \begin{block}{Bare mental or ELF}
        \begin{itemize}
        \item breakpoint, software single step, disasembly,
        \item inferior call, \texttt{dwarf2\_frame\_init\_reg},
        \item prologue analyzer, frame unwinding by prologue, skip prologue,
        \item epilogue analysis/detection
        \item register type, register group
        \item longjmp target

        \end{itemize}
      \end{block}

    \end{overprint}


    \column{5cm}
    \begin{overprint}

      \begin{block}{on Linux or ucLinux}
        \begin{itemize}
        \item stub prologue analyzer,
        \item signal trampoline unwinding,
        \item next pc of syscall (rt\_sigreturn, sigreturn)
        \item thread local storage
        \end{itemize}
      \end{block}

    \end{overprint}
  \end{columns}
\end{frame}

\section{Bare mental or ELF}

\subsection{breakpoint}
\frame{
  \frametitle{breakpoint}
  \begin{itemize}
  \item A special instruction or an illegal instruction
  \end{itemize}

%  \lstinputlisting[language=C,caption={simple-debugger.c},
%  firstline=10,lastline=36,
%  captionpos={b},label={listing:introduction:simple-debugger}]
%  {../../book/introduction/simple-debugger.c}

}

\subsection{software single step}
\frame{
  \frametitle{software single step}
  \begin{itemize}
  \item Insert breakpoint at the \textbf{next} instruction of \texttt{pc}
  \item Decode every instruction, and compute the \textbf{next} instruction of \texttt{pc}
  \end{itemize}
}
    
\subsection{inferior call}
\frame{
  \frametitle{inferior call}

  \begin{columns}
    \column{5cm}
    \begin{overprint}

      \begin{block}{Widely used}
        \begin{itemize}
        \item GDB 主动在target运行target函数
        \item \texttt{p func1()}
        \item allocate memory on target (call \texttt{malloc})
        \end{itemize}
      \end{block}

    \end{overprint}

    \column{5cm}
    \begin{overprint}

      \begin{block}{GDB需要知道}
        \begin{itemize}
        \item 函数的参数类型和返回类型
        \item 每个参数应该放在的位置 (寄存器 还是 stack)
        \item 返回值的位置 (寄存器 还是 stack)
        \item pass by value or pass by reference? \texttt{struct}
        \item alignment on passing stack
        \item \texttt{vararg}
        \end{itemize}
      \end{block}

    \end{overprint}
  \end{columns}
}

\frame{
  \frametitle{inferior call in gdb}
  \begin{table}
    \begin{tabular}{|l|l|l|} \hline
      GDB hook&输入&作用\\ \hline
      \tiny\textbf{arch}\texttt{\_push\_dummy\_call}&\tiny type parameter and return.&\tiny 根据ABI把参数放到应该放的地方\\&\tiny param value&\\ \hline

      \tiny\textbf{arch}\texttt{\_return\_value}&\tiny 返回值类型&\tiny 根据ABI从正确位置取得返回值\\ \hline

      \tiny\textbf{arch}\texttt{\_frame\_align}&&\tiny frame alignment\\ \hline

      \tiny\textbf{arch}\texttt{\_value\_to\_register}&& \tiny 寄存器和\texttt{Value}之间的转换\\
      \tiny\textbf{arch}\texttt{\_register\_to\_value}&&\\ \hline
      \tiny\textbf{arch}\texttt{\_push\_dummy\_id} &&\tiny 建立一个dummy \texttt{frame id}\\ \hline
    \end{tabular}
  \end{table}

}

\subsection{prologue analyzer}
\frame{
  \frametitle{prologue analyzer}
  
  prologue analyzer 干了两件事情
  \begin{table}
    \begin{tabular}{|l|l|} \hline
      \tiny prologue 在哪里结束?& \tiny 函数体的起始位置。\\ \hline
      \tiny prologue 都干了写什么?& \tiny 寄存器保存在哪里,特别是返回地址,\texttt{sp}, \texttt{fp} 保存在哪里\\ \hline
    \end{tabular}
  \end{table}

  prologue analyzer 的作用
  \begin{itemize}
  \item 如果有\textbf{DWARF}信息,GDB会用\textbf{DWARF}进行 frame unwinding (主要是 \texttt{.debug\_frame} section)
  \item 在没有\textbf{DWARF}信息的情况,GDB会分析函数的prologue,来进行frame unwinding。
  \item 为了正确解释\textbf{DWARF}的指令,GDB也需要知道函数体的起始位置。比如 \texttt{break foo}的时候。\textbf{arch}\texttt{\_skip\_prologue}。
  \end{itemize}
}

\frame{
  \frametitle{frame unwinding by prologue}

  \begin{itemize}
  \item 在知道了在prologue中,寄存器都保存在了什么位置,
  \item unwind frame的时候,用frame N的寄存器内容,去计算frame N+1 的寄存器内容
  \end{itemize}

}

\subsection{epilogue detection}
\frame{
  \frametitle{epilogue detection}

  \begin{itemize}
    \item \tiny 当使用watchpoint监视一个local variable的时候,gdb需要注意scope。(如果是全局变量,不需要检查scope的)
    \item \tiny 当程序运行到epilogue的时候,这个时候没有办法知道epilogue 代码的scope,gdb就会错误的把watchpoint删除
    \item \tiny 通过 \textbf{arch}\texttt{\_in\_function\_epilogue\_p} 知道当前程序是否在epilogue。输入就是\texttt{pc},通过分析指令,看是否是在epilogue。
  \end{itemize}


  \lstinputlisting[language=C,caption={epilogue},captionpos={b},
  label={listing:epilogue}]
  {code/in_function_epilogue_p.c}
}


\subsection{longjmp}
\frame{
  \frametitle{\texttt{longjmp}}

  \begin{itemize}
    \item \tiny \texttt{longjmp} 应该跳回到\texttt{setjmp} 的位置,但是肯定不是一条指令实现的
    \item \tiny GDB需要知道它们之间的协议,正确的让程序停止在\texttt{setjmp} 的位置
    \item \tiny \textbf{arch}\texttt{\_get\_longjmp\_target}
  \end{itemize}

  \lstinputlisting[language=C,caption={foo\_get\_longjmp\_target},captionpos={b},
  label={listing:longjmp}]
  {code/longjmp_target.c}
}
\section{Linux or ucLinux}

\subsection{stub prologue analyzer}
\frame{
  \frametitle{stub prologue analyzer}

  \begin{overprint}

    \begin{block}{Background}
      \begin{itemize}
      \item 在调用共享库函数的时候需要有\texttt{.plt .got}的支持
      \item \texttt{bl malloc@plt}.  Every time, in plt stub, code loads target address of \texttt{malloc}, and then jump to it.
      \item For the first time, \texttt{dynamic linker} is loaded to resolves the address of \texttt{malloc}, and update target address of \texttt{malloc}
      \end{itemize}
    \end{block}

    \begin{block}{GDB doesn't show details above}
      \begin{itemize}
       \item 当用\texttt{next}命令,GDB会让程序运行到下一行,但是会跨过函数调用。
       \item 当程序在plt stub的时候,GDB必须知道当前程序在 inner frame。GDB不知道的话,就会让程序停下来。
       \item GDB需要有一个特殊的frame unwinder for stub
      \end{itemize}
    \end{block}

  \end{overprint}

}

\frame{
  \frametitle{stub frame unwinder}
  \lstinputlisting[language=C,caption={stub frame unwinder},captionpos={b},
  label={listing:stub}]
  {code/stub_frame_unwinder.c}
}


\subsection{signal trampoline frame unwinding}
\frame{
  \frametitle{signal trampoline frame unwinding}

  \begin{itemize}
    \item GDB已经有了很好的infrastructure对signal trampoline frame unwinding支持。
    \item 每个port需要定义和自己的signal trampoline匹配的instruction pattern。参见 \texttt{arm-linux-tdep.c:struct tramp\_frame arm\_linux\_rt\_sigreturn\_tramp\_frame}
    \item 而且还要从kernel的代码中看出,当在 signal trampoline frame的时候,\texttt{struct rt\_sigframe} 所在位置。一般都是一个基于\texttt{sp}的偏移。参见 \texttt{arm-linux-tdep.c:arm\_linux\_sigreturn\_init}
  \end{itemize}

}



\frame{
  \frametitle{}


}

\subsection{next pc of syscall}

\begin{frame}
  \frametitle{next pc of syscall}
  \begin{columns}
    \column{5cm}
    \begin{overprint}

      \begin{block}{Next pc}
        \begin{itemize}
        \item 在实现\texttt{software single step}的时候已经实现完了
        \item syscall的下一条指令地址不一定是 pc + 4,比如 \texttt{sigreturn} or \texttt{rt\_sigreturn}。
        \end{itemize}
      \end{block}

    \end{overprint}


    \column{5cm}
    \begin{overprint}

      \begin{block}{GDB中实现}
        \begin{itemize}
        \item 没有相应的GDB hook,但是MIPS和ARM有类似的实现。
        \item 在计算 next pc的函数中,考虑系统调用指令。查看系统调用的规范,知道系统调用号,如果是\texttt{sigreturn} or \texttt{rt\_sigreturn},从规范指定的寄存器中读出下一条指令的地址。
        \end{itemize}
      \end{block}

    \end{overprint}
  \end{columns}
\end{frame}

\subsection{thread local storage}
\frame{
  \frametitle{thread local storage}
}


\section{End}
\frame{
  \frametitle{End}
}

\end{CJK}
\end{document}
